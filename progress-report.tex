\documentclass{article}

\usepackage{amsmath}
%\usepackage{cite}
%\usepackage{hyperref}
\usepackage[a4paper,margin=1in]{geometry}
\usepackage{enumitem}

\begin{document}
\title{Progress report -- Formalising a semiring framework for shortest-distance problems}
\author{James Wood -- jdw74@cam.ac.uk}
\maketitle

\begin{itemize}
  \item Supervisors:
    \begin{itemize}
      \item Dominic Mulligan (dominic.p.mulligan@googlemail.com)
      \item Matthew Daggitt (mld46@cam.ac.uk)
    \end{itemize}
  \item Director of Studies: Alastair Beresford (arb33@cam.ac.uk)
  \item Overseers
    \begin{itemize}
      \item Robert Watson (robert.watson@cl.cam.ac.uk)
      \item Hatice Gunes (hatice.gunes@cl.cam.ac.uk)
    \end{itemize}
\end{itemize}

\section{Introduction}
The intent of this project was to reproduce the algorithm and proofs given in \cite{Mohri02} in the proof assistant Agda\cite{Norell07}.
The original aim was to cover everything up to theorem 1, establishing that the first algorithm given always terminates, and does so with the correct result.
The algorithm is essentially a generalised version of the Dijkstra and Bellman-Ford algorithms for calculating shortest distances in a weighted graph.
It is parametric in queue discipline of the queue used to hold the next vertices to be explored.
It also has particularly loose restrictions on the weights in the graph.
Specifically, if the algorithm is to be run on a graph $G$, the weights must come from a set that forms a semiring which is $k$-closed on $G$, for some natural number $k$.
A semiring is an algebraic structure on which 0, 1, addition ($\oplus$), and multiplication ($\otimes$) are well-behaved, and it is $k$-closed on $G$ if, for any possible path through $G$, going round a loop more than $k$ times does not produce shorter distances.
A simple example uses non-negative numbers extended with $\infty$, setting 0 to $\infty$, 1 to $0$, $\oplus$ to binary minimum, and $\otimes$ to $+$.

In full generality, a distance needn't correspond to any particular path.
The paper gives the example of computing the $k$ shortest distances and $k$ distinct shortest distances using the same algorithm.
These clearly don't correspond to single paths, given that a single distance summarises $k$ different distances.
This makes the algorithm difficult to reason about in an intuitive way.

\section{Progress}
The start of the project ran to schedule.
I formalised section 1 of the paper, dealing with definitions and proofs about semirings.
I then formalised some of the notions that would be needed for working with graphs, including most of the definitions in section 2.
After this, I found a way of expressing the algorithm given in section 3, and started proving lemmas about it.

The lemmas about the algorithm took longer than expected to prove.
One reason for this was that the proofs given in the paper sometimes make reference to state kept track of implicitly by the reader, like the number of times a vertex has been extracted from the queue so far.
It took me some time to work out how to incorporate this into my formalisation.
Over the Christmas vacation, I only managed to prove one of the five remaining lemmas, after some significant rewriting of the algorithm.
I had planned to prove all of them.

Following this, my plan for the next two weeks is to focus on proving the main theorem, under the assumptions that the lemmas hold.
This theorem establishes that the algorithm, when run with all the preconditions holding, always terminates, and does so with the correct answer.
I will not do any extensions, and any spare time will be spent on proving any lemmas I have assumed.
Work on the performance evaluation will be pushed back to the week beginning on 2017-02-20.
The evaluation consists of writing an unverified Haskell implementation of the algorithm (a direct translation of what I've already written), and generating/finding test graphs on which to run both of the algorithms.
When this is done, I will pick up the original plan, and start writing the dissertation.

\bibliography{bib}
\bibliographystyle{plain}

\end{document}
