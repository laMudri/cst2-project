% Target: 1512 words
Formal verification is becoming an increasingly important part of maintaining large software systems.
It has long been commonplace for compilers to verify some level of correctness of programs -- specifically, a compiler for a statically typed language will check the types of the program it is given.
If the type checker successfully checks an expression $e$ against a type $A$, we have a guarantee that when we evaluate $e$, any resulting value we get will be in a form specified by the definition of $A$.

With finer-grained types, we can have more thorough guarantees of correctness.
A type we can express in nearly all modern programming languages is the type of lists with at least one element.
We can encode this as a structure containing a value of element type and a (possibly empty) list of the remaining elements.
Using this non-empty list type justifies the use of functions which pick an element from a list, such as \texttt{head}, which picks the first element of a list.
\texttt{head} cannot be well behaved on an empty list because it has no first element to pick.
It must thus either return a \texttt{null} reference, throw an error, or explicitly state that it might not produce a value.

Dependent types are a relatively new feature to appear in programming languages, though they have a history stretching back to at least the 1970s\cite{martin-lof:aitot} and practical use in proof assistants since Coq\cite{CoqProofAssistant}, first released in 1989.
In a dependent type system, arbitrary expressions may appear in types, and thus we can make guarantees about those expressions which are checked by the type checker.
In theory, this makes programming languages with dependent types suitable for both writing programs and proving them correct.
One language aiming to do this is Agda\cite{Norell07}, which I use for the bulk of this project.

The other main component of the project is shortest distance problems.
In a shortest distance problem, we are given a weighted graph and a source vertex, and required to give, for each vertex, the shortest distance from the source to that vertex.
Simpler versions of the problem have well known solutions.
For example, when we can guarantee that all distances are positive numbers, Dijkstra's algorithm will give us shortest distances.
However, for cases where the weights are less well behaved, there is still ongoing research.

All shortest distance problems can be described by a semiring and a class of graphs whose weights are elements of the semiring.
A semiring is an algebraic structure comprising a set $\mathbb K$, elements $\bar 0$ and $\bar 1$, and binary operators $\oplus$ and $\otimes$.
$\bar 0$ is the identity for $\oplus$, and $\bar 1$ is the identity for $\otimes$.
Furthermore, both operators are associative, $\oplus$ is commutative, and $\otimes$ distributes over $\oplus$ (though this is sometimes relaxed).
The connection to shortest distance algorithms is that $\otimes$ represents composition of distances along a path, and $\oplus$ represents the choice of the shortest amongst two distances.
Then, for distances $a$ and $b$, we say that $a \leq b$ iff $a \oplus b = a$ -- that is, out of $a$ and $b$, $a$ is chosen by $\oplus$ as shortest.
In the standard setup for Dijkstra's algorithm, $\mathbb K$ (the set of weights) is $\mathbb N \cup \{ \infty \}$, $\bar 0$ (the least-chosen distance) is $\infty$, $\bar 1$ (the trivial distance) is $0$, $\oplus$ (the choice operator) is $\min$, and $\otimes$ (the composition operator) is $+$.

The problem for which I implement a solution has the distinction that the distance choice operator $\oplus$ is not selective
This means that we don't always have either $a \oplus b = a$ or $a \oplus b = b$, or equivalently $a \leq b$ or $b \leq a$, as we would have if $\oplus$ were $\min$ on natural numbers.
The problem also allows for negative weights, with the condition that there is a natural number $k$ such that for any cycle $c$ in the graph, a path that goes around $c$ more than $k$ times will not improve on a path that goes around $c$ only $k$ times.
An algorithm for this is given and proven correct in a 2002 paper by Mohri\cite{mohri02}.
However, the proof is technical and unintuitive, and we would like to be surer that it is actually correct.
