\pgfplotsset{every axis legend/.append style=
  { at={(0,1)}
  , anchor=south west
  }
}

\chapter{Evaluation}

This chapter assesses the performance of the generic single-source shortest-distance algorithm under various conditions.

%\begin{tikzpicture}
%\begin{axis}[xmode=log, ymode=log]
%\addplot[scatter, only marks] table[x=time, y=alloc, col sep=comma] {data.csv};
%\end{axis}
%\end{tikzpicture}

%\begin{tikzpicture}
%\begin{axis}[xmode=log, ymode=log]
%\addplot[scatter, only marks] table[x=time, y=alloc, col sep=comma] {data/dijkstra.csv};
%\addplot[scatter, only marks] table[x=time, y=alloc, col sep=comma] {data/mohri.csv};
%\addplot[scatter, only marks] table[x=time, y=alloc, col sep=comma] {data/mohriKShortest4.csv};
%\addplot[scatter, only marks] table[x=time, y=alloc, col sep=comma] {data/mohriKShortest8.csv};
%\addplot[scatter, only marks] table[x=time, y=alloc, col sep=comma] {data/mohriKShortest12.csv};
%\addplot[scatter, only marks] table[x=time, y=alloc, col sep=comma] {data/mohriKShortest16.csv};
%\addplot[scatter, only marks] table[x=time, y=alloc, col sep=comma] {data/mohriKShortest20.csv};
%\addplot[scatter, only marks] table[x=time, y=alloc, col sep=comma] {data/mohrip.csv};
%\addplot[scatter, only marks] table[x=time, y=alloc, col sep=comma] {data/mohripKShortest4.csv};
%\addplot[scatter, only marks] table[x=time, y=alloc, col sep=comma] {data/mohripKShortest8.csv};
%\addplot[scatter, only marks] table[x=time, y=alloc, col sep=comma] {data/mohripKShortest12.csv};
%\addplot[scatter, only marks] table[x=time, y=alloc, col sep=comma] {data/mohripKShortest16.csv};
%\addplot[scatter, only marks] table[x=time, y=alloc, col sep=comma] {data/mohripKShortest20.csv};
%\end{axis}
%\end{tikzpicture}

\section{Algorithms for the classical shortest-distance problem}

This section concerns the performance of Mohri's algorithm when finding a shortest distance to each node in a graph.
I take distances to be from the tropical semiring $\mathbb N \cup \infty$, making the problem amenable for classical shortest-distance algorithms.
Thus, I use Dijkstra's algorithm as a reference point.

\newcommand{\plot}[2]{%
  \begin{subfigure}[b]{0.5\textwidth}
    \begin{tikzpicture}
      \begin{axis}[xlabel={$n$}, ylabel={#1},
        use units, y unit=\ifstrcmp{#1}{time}{s}{B}, y unit prefix=\ifstrcmp{#1}{time}{}{Gi},
        width=7.5cm]
        \addplot table[x=n, y=#1, col sep=comma] {data/graph-#2-dijkstra.csv};
        \addplot table[x=n, y=#1, col sep=comma] {data/graph-#2-mohri.csv};
        \addplot table[x=n, y=#1, col sep=comma] {data/graph-#2-mohrip.csv};
        \legend{Dijkstra, Mohri (LIFO), Mohri (Dijkstra-style queue)}
      \end{axis}
    \end{tikzpicture}
  \end{subfigure}%
}

\newcommand{\plottwo}[1]{%
  \begin{figure}
    \plot{time}{#1}
    \plot{alloc}{#1}
    \caption{Time taken (left) and memory allocated (right) in solving the shortest-distance problem on a \ifstrcmp{#1}{C}{complete randomly weighted}{\ifstrcmp{#1}{U}{complete uniformly weighted}{bimodal randomly weighted}} graph with $n$ vertices.}
    \label{fig:plot-#1}
  \end{figure}%
}

\newcommand{\plotk}[2]{%
  \begin{subfigure}[b]{0.5\textwidth}
    \begin{tikzpicture}
      \begin{axis}[xlabel={$k$}, ylabel={#1},
        use units, y unit=\ifstrcmp{#1}{time}{s}{B}, y unit prefix=\ifstrcmp{#1}{time}{}{Gi}, scaled y ticks=false,
        width=7.5cm]
        \addplot table[x=k, y=#1, col sep=comma] {data/graph-mohriKShortest-#2-400.csv};
        \addplot table[x=k, y=#1, col sep=comma] {data/graph-mohripKShortest-#2-400.csv};
        \legend{Mohri (LIFO), Mohri (Dijkstra-style queue)}
      \end{axis}
    \end{tikzpicture}
  \end{subfigure}%
}

\newcommand{\plotktwo}[1]{%
  \begin{figure}
    \plotk{time}{#1}
    \plotk{alloc}{#1}
    \caption{Time taken (left) and memory allocated (right) in solving the $k$ shortest-distances problem on a \ifstrcmp{#1}{C}{complete randomly weighted}{\ifstrcmp{#1}{U}{complete uniformly weighted}{bimodal randomly weighted}} graph with 400 vertices.}
    \label{fig:plot-#1}
  \end{figure}%
}

\subsection{Complete graphs}\label{sec:complete-graphs}

The first set of tests concern graphs in which each vertex is connected to every other vertex by an edge of finite distance.
Furthermore, distances are randomly assigned.
For each number of vertices $n$, I generate 8 separate graphs, each with edge weights sampled from a different random distribution.
The time and memory allocation values plotted in Figure \ref{fig:plot-C} are each a mean of the respective statistic from each of these 8 graphs.

Each of the random distributions is geometric, with the parameter set to $1/2^i$ for $i$ from 1 to 8.
I use the definition of the geometric distribution that includes 0, so given fixed parameter $p \in (0,1]$, $p$ of the sampled values are 0, $p$ of the remainder are 1, and so on.
The effect of the parameter on the performance of the algorithms will be investigated later.

The data show that, with little useful structure in a highly connected graph, having a sophisticated queueing discipline provides only minor benefits over a simple LIFO stack.

\plottwo{C}

\subsection{Uniform-weighted complete graphs}

In this set of tests, each vertex is connected to each vertex (including itself) by an edge of weight 1.
The results for these are shown in Figure \ref{fig:plot-U}.
Given that there is no parameter to vary, each data point of these is the result of a single test.

The data show that the queueing discipline for Mohri's algorithm has basically no effect on performance.
This makes sense when we consider that no queueing discipline can distinguish between the vertices in a helpful way.
It is, to an extent, surprising that the overhead costs of maintaining a priority queue are minimal.
%This suggests that the cost of the semiring operations is dominating the running time and memory allocation.

\plottwo{U}

\subsection{Bimodal graphs}

The final kind of graphs I test is graphs where vertices are split into two separate groups.
Within each group, each vertex is connected to each vertex by an edge of random weight (from the same distributions explained in Section \ref{sec:complete-graphs}).
Between the groups, there is only a single edge, from one vertex in the first group to one vertex in the second group.
The non-existence of an edge is represented by an edge of weight $\infty$.
This matches the Agda formalisation, where a graph is defined to be a function from pairs of vertices to weights.

The results for these tests is shown in Figure \ref{fig:plot-B}, with the same format as the data in Section \ref{sec:complete-graphs}.
Here, we can see that the extra structure of bimodal graphs over complete graphs gives a large advantage to the priority queues of Dijkstra's algorithm and the corresponding specialisation of Mohri's algorithm.

\plottwo{B}

\subsection{Effect of the probability distribution}

\section{$k$ shortest distances}

\subsection{Complete graphs}

\plotktwo{C}

\subsection{Uniform-weighted complete graphs}

\plotktwo{U}

\subsection{Bimodal graphs}

\plotktwo{B}
