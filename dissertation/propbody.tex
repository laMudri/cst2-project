\section*{Introduction}
Dijkstra's algorithm and the Bellman-Ford algorithm are two solutions to the single-source shortest-distance problem.
It has been long understood that classical single-source shortest-distance algorithms based on the \emph{relaxation} technique can be generalised so that edge weights are elements of an arbitrary semiring with some restrictions\cite{ahl74}.
Mohri, in \cite{Mohri02}, shows that the restrictions can be weakened to \emph{$k$-closedness on the graph} on which the algorithm will be run.
Intuitively, this allows for negative edge weights as long as the graph contains no negative-weight cycles.
Mohri puts the generality of the semiring used to use in solving the $k$-shortest-distance and $k$-distinct-shortest-distance problems, which would be challenging to solve otherwise.
The algorithm presented in Mohri's paper is also generic in the discipline of the queue of new vertices to be explored, with shortest-first queueing giving Dijkstra's algorithm and first-in first-out queueing giving the Bellman-Ford algorithm.

Formal verification is a subfield of computer science that has been researched since the 1970s\cite{DBLP:conf/aisc/2008}, starting with the Mizar proof assistant\cite{Naumowicz2009}.
There have been several important results in recent years, including proofs of the four colour theorem\cite{4-colour} and Kepler conjecture\cite{Hales-Kepler}, and verification of a C compiler\cite{Leroy-Compcert-CACM} and an ML compiler\cite{CakeML}.
The basic idea of a proof assistant is that the user writes, in some formal language(s), a mathematical statement and a proposed proof of that statement.
The proof assistant will then check whether the proof is well-formed, and if so, the statement is considered proven.
We usually want this checking process to be at least semi-decidable (all well-formed proofs get accepted by the checker), if not decidable.
In reality, practical proof assistants offer some level of interactivity, allowing the user to check small fragments of a proof rather than giving a whole proof in one go.
This means that the terms \emph{proof assistant} and \emph{interactive theorem prover} are mostly synonymous.

Agda is a dependently typed programming language and proof assistant, with the current version, Agda 2, first appearing in 2007\cite{Norell07}.
It aims to offer a single language for writing both programs and proofs, as justified by the Curry-Howard correspondence\cite{curry1980h}.
This distinguishes it from tactics-driven proof assistants, such as HOL and Coq, where programs are written in a functional programming style, but proofs are in the form of scripts comprising instructions to manipulate a proof state (tactics).
Agda is somewhat experimental, and it is often used for prototyping new ideas for dependently typed functional programming -- such as coinduction and static reflection.

In this project, I will formalise some of the paper by Mohri, at least up to and including theorem 1, which states that the first single-source shortest-distance algorithm presented is correct.
The work will be done in Agda.

\section*{Work that has to be done}

\begin{enumerate}
  \item
    Learn and practise enough Agda to complete the project.
  \item
    Read and thoroughly understand Mohri's paper.
    The remaining objectives pertain to this paper, and are broken down by section.
  \item
    Prove the required lemmas about semirings from section 1.
    Agda's standard library\footnote{\url{https://github.com/agda/agda-stdlib}} contains a definition of \emph{semiring} and \emph{commutative semiring} which match those in Mohri's paper, except that the standard library definition makes explicit reference to the equivalence relation being used.
    I will stick to the standard library definition to make my code work better with other people's and be more idiomatic.
  \item
    Make further definitions specific to shortest-distance problems.
    Some of these (like a definition of graph) are assumed by the paper, whilst others are given explicit definitions.
    These are the following.
    \begin{enumerate}
      \item
        Create a suitable definition of a weighted graph which is both easy to reason about and efficient to compute with.
      \item
        Define a \emph{path} on such a graph.
      \item
        Define the weight of a path, and the weight of a finite set of paths.
      \item
        Define shortest distance between two points.
      \item
        Define the notion of \emph{cycle} in a graph.
      \item
        Define $k$-closedness with respect to a graph.
      \item
        Define abstractly how a queue should behave.
    \end{enumerate}
  \item
    Reproduce the \textsc{Generic-Single-Source-Shortest-Distance} algorithm and proofs about it in section 3.
    The algorithm needs to be translated from an imperative-iterative style to a functional-recursive style.
    In Agda, this definition will require a proof of termination, and so this step will require doing the work of later lemmas earlier than given in the paper.
  \item
    Test the performance of the implementation (with the executable extracted using the compiler back-end MAlonzo) on some real-world data, such as a street map.
    This will require having an example semiring and several example queue structures.
    I will also produce an unverified version in Haskell to test against.
    I choose Haskell here because MAlonzo produces Haskell code, so the comparison is reasonable.
\end{enumerate}

\section*{Starting Point}
I have some experience with Agda, having used it for the past year.
I have also had an initial read through of Mohri's paper.
Having taken the IA Algorithms course, I know the problem being solved and two classical solutions to it.
My understanding of mathematical conventions for algebraic structures and proofs comes from a combination of my interests and having taken the IA Mathematical courses Vectors and Matrices and Analysis I.

Agda has a standard library, containing definitions for semirings, commutative semirings, and partial orders, which I will be using.
I will also use other things from the standard library, such as lists and lemmas about them.

\section*{Success Criteria}
\begin{enumerate}
  \item
    Read thoroughly the paper by Mohri, and understand the proofs.
  \item
    Learn and practise using Agda well enough to complete the project.
  \item
    Formalise and prove all of the lemmas about semirings from section 1 of the paper.
  \item
    Do all of the preparatory work from section 2, linking semirings with weighted graphs.
  \item
    Translate the algorithm given in the paper from imperative to functional style, proving termination in the process.
  \item
    Prove the correctness of the translated algorithm.
  \item
    Produce example semirings and queueing disciplines.
  \item
    Find example dataset (a real-world road map, for example) and manipulate it so that the algorithm can be run on it.
  \item
    Write an unverified implementation of the algorithm and run it on the example data.
\end{enumerate}

\section*{Potential Extensions}
\begin{enumerate}
  \item
    Formalise the \textsc{Generic-Topological-Single-Source-Shortest-Distance} algorithm from section 4.
  \item
    Investigate whether the complexity results in the paper are achieved by the translated algorithm.
  \item
    Prove theorems 3 and 4 about the $k$-shortest-distance and $k$-distinct-shortest-distance problems.
\end{enumerate}

\section*{Resources}
I will need a lab computer with on which to run the Agda compiler and its associated Emacs mode.
It will need plenty of RAM and a good processor.
This is required because of Agda's memory usage and slow speed for compiling large projects.
Compilation time is important because the standard workflow in using Agda relies heavily on the compiler.
Typically, the vast majority of expressions entered will be checked by the type checker before becoming part of the source code proper.

I will store all source code and writing in a Git-controlled repository.
This will be shared between my personal laptop and the lab machine via GitHub, which also serves as a back-up.

\section*{Work Plan}
The planned starting date is 2016-10-17.

% Preparation
\subsection*{Weeks 1 and 2 (2016-10-17 to 10-28)}
Understand the proofs in the paper. Set up the lab machine. Do further Agda practice. Review functional graph implementations in preparation for formalising section 2.

% Implementation
\subsection*{Weeks 3 and 4 (10-31 to 11-11)}
Make all of the definitions and proofs from section 1 except definition 7, developing semiring theory.

\subsection*{Weeks 5 and 6 (11-14 to 11-25)}
Make definition 7.
The difficulty with this definition is that it requires a notion of infinite sums not defined in the paper.
Though this definition (closedness of a semiring) is not used directly, infinite sums are used soon, so this gives a good opportunity to develop infinite sums without worrying about paths in graphs.
It will also be helpful to have when producing an example semiring to use in the evaluation phase of the project.

Make the definitions in section 2, as described in \emph{Work that has to be done} point 4.

\subsection*{Weeks 7 and 8 (11-28 to 12-09)}
Reproduce the algorithm.
At first, I will produce a function which computes a single step of the outer \textbf{while} loop.
Then, I will start proving some of the lemmas required to prove the termination of the whole algorithm.

\subsection*{Weeks 9 and 10 (12-12 to 2017-01-06, including 2 weeks' break)}
Finish the algorithm and work through the remaining lemmmas before theorem 1.

\subsection*{Weeks 11 and 12 (01-09 to 01-19)}
Prove the correctness part of theorem 1, that the distances computed by the \textsc{Generic-Single-Source-Shortest-Distance} algorithm are correct.

\subsection*{Weeks 13 and 14 (01-23 to 02-03)}
Find data to be used for the evaluation.
Simplify the dataset, and present it in formats suitable for Agda and Haskell programs.
Make a way of reading the data in Agda.
Write an unverified implementation of the program in Haskell which does the same things.

\subsection*{Weeks 15 and 16 (02-06 to 02-17)}
Do any of the potential extensions, depending on the amount of time left.
Extension 2, concerning the complexity results, would probably be the quickest, given that it would not be formalised.
Both of the other extensions involve a moderate amount of extra work.
Extension 1, about the \textsc{Generic-Topological-Single-Source-Shortest-Distance} algorithm is more algorithmic and extension 3, about $k$-shortest-distances is more algebraic.
I will be able to judge between these better after having done the main part of the project.

% Writing
\subsection*{Weeks 17 and 18 (02-20 to 03-03)}
Start writing the dissertation.
Make a skeleton document containing the required sections.
Finish the preparation section and most of the implementation section.

\subsection*{Weeks 19 and 20 (03-06 to 03-17)}
Finish off the implementation section.
Write the evaluation section.
Act upon feedback to draft sections as they come.

\subsection*{Easter break (03-20 to 04-21)}
Write the introduction and conclusions sections.
Produce a full draft.
Iteratively get feedback and edit the draft to get it ready for submission.
